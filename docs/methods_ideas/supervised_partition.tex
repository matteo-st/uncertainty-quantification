\documentclass{article}
\usepackage{amsmath,amssymb,amsthm}
\usepackage{algorithm}
\usepackage{algpseudocode}
\usepackage[margin=1in]{geometry}

\begin{document}

\section*{Option 3: Risk-aware balanced recursive partition into axis-aligned rectangles (2D)}

\paragraph{Goal.}
Given calibration data
\[
\mathcal D = \{(u_{1,k},u_{2,k},E_k)\}_{k=1}^n,\qquad (u_{1,k},u_{2,k})\in[0,1]^2,\ E_k\in\{0,1\},
\]
construct $B$ axis-aligned rectangles (leaf regions) with \emph{near-uniform mass} (roughly $n/B$ points per leaf),
where splits are chosen to \emph{separate error rates} between children.

\paragraph{Node representation.}
A node (region) $\nu$ stores:
\begin{itemize}
  \item an index set $S_\nu \subseteq [n]$ of points currently in the node,
  \item a bounding box $R_\nu = [\ell_{1,\nu}, r_{1,\nu}) \times [\ell_{2,\nu}, r_{2,\nu}) \subseteq [0,1]^2$,
  \item the empirical error rate $\widehat\eta(\nu) \coloneqq \frac{1}{|S_\nu|}\sum_{k\in S_\nu} E_k$.
\end{itemize}
The root node is $\nu_0$ with $S_{\nu_0}=[n]$ and $R_{\nu_0}=[0,1)\times[0,1)$.

\paragraph{Size constraint (near-uniform mass).}
Fix
\[
n_{\min} \coloneqq \left\lfloor \frac{n}{B} \right\rfloor \quad (\text{e.g., } n=4000,\ B=30 \Rightarrow n_{\min}\approx 133).
\]
Every split must produce children with at least $n_{\min}$ points.

\paragraph{Candidate thresholds.}
Fix a small set of quantile levels
\[
\mathcal Q = \{q_1,\dots,q_M\}\subset(0,1)\quad \text{(e.g., } \{0.1,0.2,\dots,0.9\}\text{)}.
\]
For a node $\nu$ and dimension $d\in\{1,2\}$, define candidate thresholds
\[
\tau \in \Big\{ \mathrm{Quantile}_{q}( \{u_{d,k}:k\in S_\nu\} ) : q\in\mathcal Q \Big\},
\]
optionally discarding duplicates (ties).

\paragraph{Split and score.}
For node $\nu$, a candidate split $(d,\tau)$ induces children:
\[
S_L=\{k\in S_\nu: u_{d,k}\le \tau\}, \qquad S_R = S_\nu\setminus S_L,
\]
with corresponding rectangles (updating only the split coordinate bounds):
\[
R_L = R_\nu \cap \{u_d \le \tau\},\qquad R_R = R_\nu \cap \{u_d > \tau\}.
\]
The split is \emph{admissible} if $|S_L|\ge n_{\min}$ and $|S_R|\ge n_{\min}$.

Define empirical child error rates
\[
\widehat\eta_L=\frac{1}{|S_L|}\sum_{k\in S_L}E_k,\qquad
\widehat\eta_R=\frac{1}{|S_R|}\sum_{k\in S_R}E_k.
\]
A simple, effective split score is the absolute separation:
\[
\Delta(d,\tau;\nu)\coloneqq |\widehat\eta_L - \widehat\eta_R|.
\]
Optionally (to discourage very unbalanced splits even when admissible), use
\[
\Delta'(d,\tau;\nu) \coloneqq \Delta(d,\tau;\nu)\cdot \sqrt{\frac{|S_L||S_R|}{|S_\nu|^2}}.
\]
(Use either $\Delta$ or $\Delta'$ consistently.)

\paragraph{Tree growth strategy (exactly $B$ leaves).}
Maintain a set of current leaves $\mathcal L$ (initially $\{\nu_0\}$). Repeatedly split \emph{one} current leaf at a time until $|\mathcal L|=B$.
At each iteration, among all leaves, select the leaf whose \emph{best admissible split} has the largest score.

\begin{algorithm}[h]
\caption{Risk-aware balanced rectangle partition (2D)}
\begin{algorithmic}[1]
\Require Data $\mathcal D=\{(u_{1,k},u_{2,k},E_k)\}_{k=1}^n$, target leaves $B$, quantiles $\mathcal Q$, minimum leaf size $n_{\min}$.
\Ensure A binary tree; leaves define rectangles $\{R_\nu\}_{\nu\in\mathcal L}$, with associated $S_\nu$ and $\widehat\eta(\nu)$.
\State Initialize root leaf $\nu_0$ with $S_{\nu_0}=[n]$, $R_{\nu_0}=[0,1)\times[0,1)$.
\State $\mathcal L \gets \{\nu_0\}$.
\While{$|\mathcal L|<B$}
  \State $bestLeaf \gets \text{None}$; $bestSplit \gets \text{None}$; $bestScore \gets -\infty$.
  \ForAll{leaves $\nu\in\mathcal L$}
    \State Compute the best admissible split $(d^\star,\tau^\star)$ for $\nu$:
      \Statex\quad $score^\star(\nu) \gets \max\{\Delta(d,\tau;\nu): d\in\{1,2\},\ \tau\in \mathcal T_{d,\nu},\ \text{admissible}\}$
      \Statex\quad where $\mathcal T_{d,\nu}$ are candidate thresholds from quantiles in $\mathcal Q$.
    \If{no admissible split exists for $\nu$} \State continue \EndIf
    \If{$score^\star(\nu) > bestScore$}
      \State $bestScore \gets score^\star(\nu)$
      \State $bestLeaf \gets \nu$
      \State $bestSplit \gets (d^\star,\tau^\star)$
    \EndIf
  \EndFor
  \If{$bestLeaf$ is None} \Comment{no admissible splits anywhere}
    \State \textbf{break}
  \EndIf
  \State Split $bestLeaf$ using $bestSplit$ into children $\nu_L,\nu_R$:
  \Statex\quad $S_{\nu_L}=\{k\in S_{bestLeaf}: u_{d^\star,k}\le \tau^\star\}$, $S_{\nu_R}=S_{bestLeaf}\setminus S_{\nu_L}$
  \Statex\quad $R_{\nu_L}=R_{bestLeaf}\cap\{u_{d^\star}\le \tau^\star\}$,\ \ $R_{\nu_R}=R_{bestLeaf}\cap\{u_{d^\star}>\tau^\star\}$
  \State Update leaf set: $\mathcal L \gets (\mathcal L\setminus\{bestLeaf\})\cup\{\nu_L,\nu_R\}$.
\EndWhile
\end{algorithmic}
\end{algorithm}

\paragraph{Output.}
The final leaves $\mathcal L$ define $|\mathcal L|\le B$ rectangles $\{R_\nu\}$.
For each leaf, store:
\[
R_\nu = [\ell_{1,\nu}, r_{1,\nu}) \times [\ell_{2,\nu}, r_{2,\nu}),\qquad
\widehat\eta(\nu) = \frac{1}{|S_\nu|}\sum_{k\in S_\nu}E_k,\qquad |S_\nu|.
\]
In downstream calibration/certification, treat $Q(u_1,u_2)=\nu$ if $(u_1,u_2)\in R_\nu$.

\paragraph{Inference-time assignment.}
Given a new point $(u_1,u_2)$, traverse the tree from the root:
at each internal node with split $(d,\tau)$ go left if $u_d\le \tau$ and right otherwise, until a leaf $\nu$ is reached.
Return $Q(u_1,u_2)=\nu$.

\paragraph{Implementation notes (important for correctness).}
\begin{itemize}
  \item \textbf{Ties at thresholds.} Use the convention ``left if $u_d\le\tau$'' consistently in both training and inference.
  \item \textbf{Quantile computation.} Quantiles are empirical over $\{u_{d,k}:k\in S_\nu\}$; remove duplicates to avoid empty children.
  \item \textbf{Exact $B$ leaves.} The loop increases leaf count by $+1$ each split. If no admissible split exists before reaching $B$, you stop early.
  \item \textbf{Uniformity.} This enforces a \emph{lower bound} $n_{\min}$; if you want tighter near-uniformity, also enforce an upper bound
        $|S_{\text{child}}|\le n_{\max}$ (e.g.\ $n_{\max}\approx \lceil 1.3\,n/B\rceil$) by discarding splits that create too-large leaves.
\end{itemize}

\end{document}
